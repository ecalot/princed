
                             Prince of Persia
                               File Formats
                              Specifications

Table of Contents
~~~~~ ~~ ~~~~~~~~
1. Preamble ............................................................ 53
2. Introduction ........................................................ 62
3. Primitives .......................................................... 87
3.1. DAT reading and writing primitives ................................ 94
3.2. DAT reading primitives ........................................... 106
3.3. DAT writing primitives ........................................... 115
4. DAT v1.0 Format Specifications ..................................... 124
4.1. General file specs, index and checksums .......................... 127
4.2. Images ........................................................... 219
4.2.1 Headers ......................................................... 223
4.2.2 Algorithms ...................................................... 248
4.2.2.1 Run length encoding (RLE) ..................................... 264
4.2.2.2 LZ variant (LZG) .............................................. 276
4.3. Palettes ......................................................... 334
4.4. Levels ........................................................... 341
4.4.1 Unknown blocks .................................................. 372
4.4.2 Room mapping .................................................... 390
4.4.2.1 Wall drawing algorithm ........................................ 526
4.4.3 Room linking .................................................... 594
4.4.4 Guard handling .................................................. 610
4.4.5 Starting Position ............................................... 657
4.4.6 Door events ..................................................... 671
4.5. Digital Waves .................................................... 715
4.6. Midi music ....................................................... 732
4.7. Internal PC Speaker .............................................. 735
4.8. Binary files ..................................................... 740
5. DAT v2.0 Format Specifications ..................................... 747
5.1. General file specs, index and checksums .......................... 750
5.1.1 The master index ................................................ 777
5.1.2 The slave indexes ............................................... 823
5.2. Levels ........................................................... 843
5.2.1 Room mapping .................................................... 867
5.2.2 Door events ..................................................... 960
5.2.3 Guard handling .................................................. 981
5.2.3.1 Static guards ................................................. 994
5.2.3.2 Dynamic guards ............................................... 1030
6. PLV v1.0 Format Specifications .................................... 1047
6.1. User data ....................................................... 1074
6.2. Allowed Date format ............................................. 1104
7. The SAV v1.0 format ............................................... 1118
8. The HOF v1.0 format ............................................... 1164
9. Credits ........................................................... 1187
10. License .......................................................... 1208


1. Preamble
   ~~~~~~~~

 This file was written thanks to the hard work on reverse engineering made
 by several people, see the credits section. In case you find any mistake
 in the text please report it. A copy of this document should be available
 in our official site at http://www.princed.com.ar.


2. Introduction
   ~~~~~~~~~~~~

 There are two versions of the DAT file format: DAT v1.0 used in POP 1.x
 and DAT v2.0 used in POP 2. In this document we will specify DAT v1.0.

 DAT files were made to store levels, images, palettes, wave, midi and
 internal speaker sounds. Each type has its own format as described below
 in the following sections.

 As the format is very old and the original game was distributed in disks,
 it is normal to think that the file format uses some kind of checksum
 validation to detect file corruptions.

 DAT files are indexed, this means that there is an index and you can
 access each resource through an ID that is unique for the resource inside
 the file.

 Images store their height and width but not their palette, so the palette
 is another resource and must be shared by a group of images.

 PLV files use the extension defined to support a format with only one
 level inside.


3. Primitives
   ~~~~~~~~~~

 This section shows how the PR dat handling primitives works, this library
 is useful to access resources without having to worry about the format.
 Here you can find the primitive chart of the dat.h library.

3.1. DAT reading and writing primitives
 Opening a dat file for RW mode
 Syntax:
 int mRWBeginDatFile(
  const char* vFile, /* the name of the file to be open */
  unsigned short int *numberOfItems, /* saves the total items count */
  int optionflag /* see optionflag appendix */
 );
 Return values are:

 int mRWCloseDatFile(dontSave);

3.2. DAT reading primitives
 int  mReadBeginDatFile(unsigned short int *numberOfItems,
      const char* vFile);
 int  mReadFileInDatFile(int indexNumber,unsigned char** data,
      unsigned long int *size);
 int  mReadInitResource(tResource** res,const unsigned char* data,
      long size);
 void mReadCloseDatFile();

3.3. DAT writing primitives
 int  mWriteBeginDatFile(const char* vFile, int optionflag);
 void mWriteFileInDatFile(const unsigned char* data, int size);
 void mWriteFileInDatFileIgnoreChecksum(unsigned char* data,int size);
 void mWriteInitResource(tResource** res);
 void mWriteCloseDatFile(tResource* r[],int dontSave,int optionflag, const
      char* backupExtension);


4. DAT v1.0 Format Specifications
   ~~~ ~~~~ ~~~~~~ ~~~~~~~~~~~~~~

4.1. General file specs, index and checksums
 All DAT files have an index, this index has a number of items count and
 a list of items.
 The index is stored at the very end of the file.
 The first 6 bytes are reserved to locate the index and know the file size.

 Let's define the numbers as:
  SC - Signed char: 8 bits, the first bit is for the sign and the 7 last
       for the number. If the first bit is a 0, then the number is
       positive, if not the number is negative, in that case invert all
       bits and add 1 to get the positive number.
       i.e. -1 is FF (1111 1111), 1 is 01 (0000 0001)
       Range: -128 to 127
       1 byte
  UC - Unsigned char: 8 bits that represent the number.
       i.e. 32 is 20 (0010 0000)
       Range: 0 to 255
       1 byte
  US - Unsigned Short: Little endian, 16 bits, storing two groups of 8 bits
       ordered from the less representative to the most representative
       without sign.
       i.e. 65534 is FFFE in hex and is stored FE FF (1111 1110  1111 1111)
       Range: 0 to 65535
       2 bytes
  SS - Signed Short: Little endian, 16 bits, storing two groups of 8 bits
       ordered from the less representative to the most representative with
       sign. If the first byte is 0 then the number is positive, if not the
       number is negative, in that case invert all bits and add 1 to get
       the positive number.
       i.e. -2 is FFFE in hex and is stored FE FF (1111 1110  1111 1111)
       Range: -32768 to 32767
       2 bytes
  UL - Unsigned long: Little endian, 32 bits, storing four groups of 8 bits
       each ordered from the less representative to the most representative
       without sign.
       i.e. 65538 is 00010002 in hex and is stored 02 00 01 00
       (0000 0010  0000 0000  0000 0001  0000 0000)
       Range: 0 to 2^32-1
       4 bytes

 Note: Sizes are always in bytes unless another unit is specified.

 Index structures:

 The DAT header: Size = 6 bytes
  - Offset 0, size 4, type UL: IndexOffset
           (the location where the index begins)
  - Offset 4, size 2, type US: IndexSize
           (the number of bytes the index has)
           Note that IndexSize is 8*numberOfItems+2
           Note that IndexOffset+IndexSize=file size

 The DAT index: Size = IndexSize bytes
  - Offset IndexOffset,   size 2, type US: NumberOfItems
           (resources count)
  - Offset IndexOffset+2, size NumberOfItems*8: The index
           (a list of NumberOfItems blocks of 8-bytes-length index record)

 The 8-bytes-length index record (one per item): Size = 8 bytes
  - Relative offset 0, size 2, type US: Item ID
  - Relative offset 2, size 4, type UL: Resource start
           (absolute offset in file)
  - Relative offset 6, size 2, type US: Size of the item
           (not including the checksum byte)

 Note:
  POP1 doesn't validate a DAT file checking:
  IndexOffset+IndexSize=FileSize
  this means you can append data at the end of the file.

  PR validates that IndexOffset+IndexSize<=FileSize.
 It also compares IndexSize with 8*numberOfItems+2 to determine if a file
  is a valid POP1 DAT file.

 Checksum byte:
 There is a checksum byte for each item (resource), this is the first byte
 of the item, the rest of the bytes are the item data. The item type is not
 stored and may only be determined by reading the data and applying some
 filters, unfortunately this method may fail. When you extract an item you
 should know what kind of item you are extracting.

 If you add (sum) the whole item data including checksum and take the less
 representative byte (modulus 256) you will get the sum of the file. This
 sum must be FF in hex (255 in UC or -1 in SC). If the sum is not FF, then
 adjust the checksum in order to set this value to the sum. The best way
 to do that is adding all the bytes in the item data (excluding the
 checksum) and inverting all the bits. The resulting byte will be the
 right checksum.

 From now on the specification are special for each data type (that means
 we won't include the checksum byte anymore).

4.2. Images
 Images are stored compressed and have a header and a compressed data area.
 Each image only one header with 6 bytes in it as follows

4.2.1 Headers
 The 6-bytes-image header: 6 bytes
  Relative offset 0, size 2, type US: Height
  Relative offset 2, size 2, type UL: Width
  Relative offset 4, size 2: Information

 Information is a set of bits where:
  the first 8 are zeros
  the next 4 are the resolution:
   if it is 1011 (B in hex) then the image has 16 colours
   if it is 0000 (0 in hex) then the image has 2 colours
   so to calculate the bits per pixel there are in the image, just take the
   last 2 bits and add 1. e. g. 11 is 4 (2^4=16 colours) and
  00 is 1 (2^1=2 colours).
  the last 4 bits are the 5 compression types:
   from 0 to 4:
   0 RAW_LR (0000)
   1 RLE_LR (0001)
   2 RLE_UD (0010)
   3 LZG_LR (0011)
   4 LZG_UD (0100)

 The following data in the resource is the image compressed with the
 algorithm specified by those 4 bits.

4.2.2 Algorithms
 RAW_LR means that the data wasn't compressed, it is used for small images.
        The format is saved from left to right (LR) serialising a line to
        the next integer byte if necessary. In case the image was 16
        colours, two pixels per byte (4bpp) will be used. In case the image
        was 2 colours, 8 pixels per byte (1bpp) will be used.
 RLE_LR has a Run length encoding (RLE) algorithm, after uncompressed the
        image can be read as a RAW_LR.
 RLE_UD is the same as RLE_LR except that after uncompressed the bytes in
     the image must be drawn from up to down and then from left to right.
 LZG_LR has some kind of variant of the LZ77 algorithm (the sliding windows
        algorithm), here we named it LZG in honour of Lance Groody, the
        original coder.
        After decompressed it may be handled as RAW_LR.
 LZG_UD Uses LZG compression but is drawn from top to bottom as RLE_UD

4.2.2.1 Run length encoding (RLE)
 The first byte is always a control byte, the format is SC. If the control
 byte is negative, then the next byte must be repeated n times as the bit
 inverted control byte says, after the next byte (the one that was
 repeated)
 another control byte is stored.
 If the control byte is positive or zero just copy textual the next n bytes
 where n is the control byte plus one. After that, the next byte is the
 following control byte.
 If you reach a control byte but the image size is passed, then you have
 completed the image.

4.2.2.2 LZ variant (LZG)
 This is a simplified algorithm explanation:

 Definition: "print" means to commit a byte into the current location
             of the output stream.

 The output stream is a slide window initialised with zeros.
 The first byte of the input stream is a maskbyte.
 For each of the 8 bits in the maskbyte the next actions must be performed:
  If the bit is 1 print the next unread byte to the slide window
  If the bit is a zero read the next two bytes as control bytes with the
  following format (RRRRRRSS SSSSSSSS):
   - 6  bits for the copy size number (R). Add 3 to this number.
        Range: 2 to 66
   - 10 bits for the slide position (S). Add 66 to this number.
        Range: 66 to 1090
   Then print in the slide window the next R bytes that are the same slide
   window starting with the S'th byte.

 After all the maskbyte is read and processed, the following input byte is
 another maskbyte. Use the same procedure to finish decompressing the file.
 Remaining unused maskbits should be zeros to validate the file.

 This is the modus operandi of the compression algorithm

 For each input byte we take a window containing the 1023 previous bytes.
 If the window goes out of bounds (ie, when the current input byte is
 before position 1024), we consider it filled with zeros.

     00000000000000000000********************************
                         ^                  ^
                    input start   current input byte
           |--------------------------------|
                    window size=1023

 The algorithm works as follows:

 While there is unread input data:
     Create a maskbyte.
     For each bit in the maskbyte (and there is still unread input data):
         Compare the following input bytes with the bytes in the window,
         and search the longest pattern that is equal to the next bytes.
         If we found a pattern of length n > 2:
             Assign 0 to the current bit of the maskbyte.
             In the next 2 bytes of the output, specify the relative
             position and length of the pattern.
             Advance output pointer by 2.
             Advance input pointer by n.
         Else:
             Assign 1 to the current bit of the maskbyte.
             Copy the current input byte in the next output byte.
             Advance output pointer by 1.
             Advance input pointer by 1.

 For a better understanding of the algorithm we strongly recommend to read
 the PR source files lzg_uncompress.c and lzg_compress.c that may be
 located at http://www.cvs.fp.princed.com.ar in the PR repository module.

4.3. Palettes
 Palettes have 100 bytes always, after 4 bytes from the beginning the
 first 16 records of 3 bytes are the VGA colours stored in the RGB-18 bits
 format (6 bits for each colour). Each colour is a number from 0 to 63.
 Remember to shift the colour bytes by two to get the colour number from 0
 to 256.

4.4. Levels
 This table has a summary of the blocks to be used in this section,
 you can refer it from the text below.

                   Table 4.1: DAT 1.0 Level blocks
                   ~~~~~~~~~~~~~~~~~~~~~~~~~~~~~~~

  Length Offset  Block Name
  ~~~~~~ ~~~~~~  ~~~~~~~~~~
  720    0       wall
  720    720     pop1_background
  256    1440    door I
  256    1696    door II
  96     1952    links
  64     2048    unknown I
  3      2112    start_position
  3      2115    unknown II
  1      2116    unknown III
  24     2119    guard_location
  24     2143    guard_direction
  24     2167    unknown IV (a)
  24     2191    unknown IV (b)
  24     2215    guard_skill
  24     2239    unknown IV (c)
  24     2263    guard_colour
  16     2287    unknown IV (d)
  2      2303    0F 09 (2319)

 All levels have a size of 2305, except in the original game, that the
 potion level has a size of 2304 (may be it was wrong trimmed).

4.4.1 Unknown blocks
 Blocks described in this section are: Unknown from I to IV.

 Unknown III and IV blocks doesn't affect the level if changed, if you find
 out what they are used to we will welcome your specification text.

 Unknown I may corrupt the level if edited.

 We believe unknown II has something to do with the start position, but we
 don't know about it.

 As unknown II were all zeros for each level in the original set, it was a
 team decision to use those bytes for format extension. If one of them is
 not the default 00 00 00 hex then the level was extended by the team.
 Those extensions are only supported by RoomShaker at this  moment. To see
 how those extensions were defined read the appendix I'll write some day.
 For the moment you may contact us if you need to know that.

4.4.2 Room mapping
 This section explains how the main walls and objects are stored. The
 blocks involved here are "wall" and "pop1_background"

 In a level you can store a maximum of 24 rooms (also called screens) of 30
 tiles each, having three stages of 10 tiles each. Screens are numbered
 from 1 to 24 (not 0 to 23) because the 0 was reserved for special cases.

 The wall and pop1_background blocks have 24 sub-blocks inside. Those
 sub-blocks have a size of 30 bytes each and has a room associated. So, for
 example, the sub-block staring in 0 corresponds to the room 1 and the
 sub-block starting in 690 corresponds to the room 24.
 It is reserved 1 byte from the wall block and one from the pop1_background
 block for each tile. To locate the appropriate tile you have to do the
 following calculation: tile=(room-1)*30+tileOffset where tileOffset is a
 number from 0 to 29 that means a tile from 0 to 9 if in the upper stage,
 from 10 to 19 if in the middle stage and 20 to 29 if in the bottom stage.
 We define this as the location format and will be used also in the start
 position.
 Always looking from the left to the right.
 So there is a wall and pop1_background byte for each tile in the level and
 this is stored this way.

 The wall part of the tile stores the main tile form according to the table
 below. Note that those are just a limited number of tiles, each code has a
 tile in the game. The tiles listed are all the ones needed to make a level
 so the missing tiles have an equivalent in this list.

 Each tile has a code id, as some codes are repeated this is how you have
 to calculate the codes. A tile in the wall part has 8 bits in this format
 rrmccccc, where rr are random bits and can be ignored. m is a modifier of
 the tile. For example modified loose floors do not fall down. The rest
 ccccc is the code of the tile tabled below. Tile names are the same as the
 ones used by RoomShaker to keep compatibility.

                   Table 4.2: Foreground Walls
                   ~~~~~~~~~~~~~~~~~~~~~~~~~~~

  Hex  Binary Group  Description
  ~~~~ ~~~~~~ ~~~~~  ~~~~~~~~~~~
  0x00 00000  free   Empty
  0x01 00001  free   Floor
  0x02 00010  spike  Spikes
  0x03 00011  none   Pillar
  0x04 00100  gate   Gate
  0x05 00101  none   Stuck Button
  0x06 00110  event  Drop Button
  0x07 00111  tapest Tapestry
  0x08 01000  none   Bottom Big-pillar
  0x09 01001  none   Top Big-pillar
  0x0A 01010  potion Potion
  0x0B 01011  none   Loose Board
  0x0C 01100  ttop   Tapestry Top
  0x0D 01101  none   Mirror
  0x0E 01110  none   Debris
  0x0F 01111  event  Raise Button
  0x10 10000  none   Exit Left
  0x11 10001  none   Exit Right
  0x12 10010  chomp  Chopper
  0x13 10011  none   Torch
  0x14 10100  wall   Wall
  0x15 10101  none   Skeleton
  0x16 10110  none   Sword
  0x17 10111  none   Balcony Left
  0x18 11000  none   Balcony Right
  0x19 11001  none   Lattice Pillar
  0x1A 11010  none   Lattice Support
  0x1B 11011  none   Small Lattice
  0x1C 11100  none   Lattice Left
  0x1D 11101  none   Lattice Right
  0x1E 11110  none   Torch with Debris
  0x1F 11111  none   Null

 The pop1_background part of the tile stores a modifier or attribute of the
 wall part of the tile. This works independently of the modifier bit in the
 code. The tile  modifier depends on the group the tile belongs which are
 wall, chomp, event, ttop, potion, tapp, gate, spike and free.
 The group event allows the 256 modifiers and will be described in 4.4.6.
 Note + means allowed for the dungeon environment, - means allowed for the
 palace environment.

                   Table 4.3: Background modifiers by group
                   ~~~~~~~~~~~~~~~~~~~~~~~~~~~~~~~~~~~~~~~~

  Group  Code Description
  ~~~~~  ~~~~ ~~~~~~~~~~~
  none   0x00 This value is used always for this group
  free   0x00 +Nothing -Blue line
  free   0x01 +Spot1   -No blue line
  free   0x02 +Spot2   -Diamond
  free   0x03 Window
  free   0xFF +Spot3   -Blue line?
  spike  0x00 Normal (allows animation)
  spike  0x01 Barely Out
  spike  0x02 Half Out
  spike  0x03 Fully Out
  spike  0x04 Fully Out
  spike  0x05 Out?
  spike  0x06 Out?
  spike  0x07 Half Out?
  spike  0x08 Barely Out?
  spike  0x09 Disabled
  gate   0x00 Closed
  gate   0x01 Open
  tapest 0x00 -With Lattice
  tapest 0x01 -Alternative Design
  tapest 0x02 -Normal
  tapest 0x03 -Black
  potion 0x00 Empty
  potion 0x01 Health point
  potion 0x02 Life
  potion 0x03 Feather Fall
  potion 0x04 Invert
  potion 0x05 Poison
  potion 0x06 Open
  ttop   0x00 -With lattice
  ttop   0x01 -Alternative design
  ttop   0x02 -Normal
  ttop   0x03 -Black
  ttop   0x04 -Black
  ttop   0x05 -With alternative design and bottom
  ttop   0x06 -With bottom
  ttop   0x07 -With window
  chomp  0x00 Normal
  chomp  0x01 Half Open
  chomp  0x02 Closed
  chomp  0x03 Partially Open
  chomp  0x04 Extra Open
  chomp  0x05 Stuck Open
  wall   0x00 +Normal  -Blue line
  wall   0x01 +Normal  -No Blue line

 Note: Some modifiers have not been tested, there may be any other unknown
       tile type we didn't discover.


4.4.2.1 Wall drawing algorithm
 This section doesn't have a direct relation with the format because it
 describes how the walls must be drawn on the room. However, as this
 information should be useful to recreate a cloned room read from the
 format we decided to include this section to the document.

 Wall drawing depends on what is in the right panel. If the right panel
 is not a wall (binary code ends in 10100) a base wall will be drawn and
 other random seed will be used. If the right panel is a wall then the main
 base wall will be drawn and the described seed will be used.

 When the base wall is drawn, the modifiers should be blitted over it.
 There are 53 different types of walls using the same base image.
 We will call the seed array to the one having the modifier information of
 those 53 panels. This array has indexes from 1 to 53 included.

 To calculate what value take from the array this calculation must be
 performed: panelInfo=seedArray[roomNumber+wallPosition]
 where panelInfo is the result modifier information that will be applied to
 the base image; seedArray is this array that will be described above as a
 table; roomNumber is the number of the room the wall is (from 1 to 24)
 and wallPosition is the position the wall is (from 0 to 29), using the
 location format specified in section 4.4.2. This means the first value is
 1 (roomNumber=1 and wallPosition=0) and the last is 53 (roomNumber=24
 and wallPosition=29).

 Modifiers affects the corners of a stone. There are three stone rows per
 wall. If the modifier is activated this corner will appear different
 (seems to be darker). Another modifier is the grey stone.

                   Table 4.4: Stone modifiers on seed position
                   ~~~~~~~~~~~~~~~~~~~~~~~~~~~~~~~~~~~~~~~~~~~
  Modifier       Seed Positions
  ~~~~~~~~       ~~~~ ~~~~~~~~~
     (First row modifiers)
  Grey stone     2, 5, 14, 17, 26, 32, 35, 50
  Left, bottom   2, 11, 36, 45
  Left, top      37
  Right, bottom  27, 33
  Right, up      4, 10, 31, 37

     (second row)
  Grey stone     none 
  Left, bottom   34, 47
  Left, top      9, 10
  Right, bottom  2, 8, 25, 35
  Right, top     6, 12, 23, 29, 39

     (third row)
  Grey stone     none 
  Left, bottom   none
  Left, top      16
  Right, bottom  none
  Right, top     none

 Another modifiers are saved in the seed too. Those modifiers are not
 boolean values, they are offsets and sizes. As each stone has a different
 size the stone separation offset is saved in the seed.
 For the first row, the stones are all the same size and the seed is not
 needed.
 For the second row we've got the first 20 values, ordered from 1 to 20. 
 position        1,2,3,4,5,6,7,8,9,10,11,12,13,14,15,16,17,18,19,20
 offsets:        5,4,3,3,1,5,4,2,1, 1, 5, 3, 2, 1, 5, 4, 3, 2, 5, 4
 separator size: 0,1,1,0,0,0,1,1,0, 0, 1, 1, 1, 0, 0, 1, 1, 1, 0, 0

 We'll be adding the next values as soon as we count the pixels ;)
 This information can be found in walls.conf file from FreePrince.

4.4.3 Room linking
 This section describes the links block.

 Each room is linked to another by each of the four sides. Each link
 is stored. There is no room mapping, just room linking.

 The links block has 24 sub-blocks of 4 bytes each. All those sub-blocks
 has its own correspondence with a room (the block starting at 0 is
 related to the room 1 and the block starting at with 92 is related to
 room 24).
 Each block of 4 bytes stores the links this room links to reserving one
 byte per each side in the order left (0), right (1), up (2), down (3).
 The number 0 is used when there is no room there.
 Cross links should be made to allow the kid passing from a room to
 another and then coming back to the same room but it's not a must.

4.4.4 Guard handling
 This section specifies the blocks: guard_location, guard_direction,
 guard_skill and guard_colour.

 Each guard section has 24 bytes, each byte of them corresponds to a room
 so byte 0 is related to room 1 and byte 23 is related to room 24.
 This room is where the guard is located. The format only allows one
 guard per room. Each block describes a property or attribute of the guard.

 The guard_location part of a guard describes where in the room the guard
 is located, this is a number from 0 to 29 if the guard is in the room or
 30 if there is no guard in this room. Other values are allowed but are
 equivalent to 30. The number from 0 to 29 is in the location format
 specified in section 4.4.2

 The guard_direction part describes where the guard looks at. If the value
 is 0, then the guard looks to the right, if the value is the hex FF (-1 or
 255) then he looks left. This is the direction format, and will be used in
 the start position too.

 The guard_skill is how the guard fights, style and lives. Note that the
 lives also depends on the level you are. Allowed numbers are from 0 to 9.

 TODO: add a skill table

 The guard_colour is the palette the guard has (see 4.8).
 The default colours are in this table:

                   Table 4.4: Default Guard colours
                   ~~~~~~~~~~~~~~~~~~~~~~~~~~~~~~~~

  Code Pants     Cape
  ~~~~ ~~~~~     ~~~~
  0x00 Light     Blue Pink
  0x01 Red       Purple
  0x02 Orange    Yellow
  0x03 Green     Yellow
  0x04 Dark Blue Beige
  0x05 Purple    Beige
  0x06 Yellow    Orange

 Other codes may generate random colours because the game is reading
 the palette from trashed memory. This may also cause a game crash.
 It should (never tested) be possible to add new colours in the guard
 palette resource (see 4.8) avoiding the crash due to this reason.


4.4.5 Starting Position
 This section describes the start_position block.

 This block stores where and how the kid starts in the level. Note that all
 level doors that are on the starting room will be closed in the moment
 the level starts.

 This block has 3 bytes.
 The first byte is the room, allowed values are from 1 to 24.
 The second byte is the location, see the section 4.4.2 for the location
 format specifications.
 The third byte is the direction, see 4.4.4 for the direction format
 specifications.

4.4.6 Door events
 This section explains how the doors are handled and specifies the blocks
 door I and II.

 First of all he have to define what an event line is in this file. An
 event line is a link to a door that will be activated. If the event was
 triggered with the action close, then the event will close the door, if
 the action was open then the event will open the door. An event line has
 also a flag to trigger the next event line or not.
 An event is defined as a list of event lines, from the first to the last.
 The last must have the trigger-next-event-line flag off. This is like a
 list of doors that performs an action.
 An event performs the action that it was called to do: open those doors or
 close them. This action is defined by the type of tile pressed.
 Each event line has an ID from 0 to 255. An event has the ID of the first
 event line in it.

 In section 4.4.2 it is explained how a door trigger is associated to an
 event ID. Those are the tiles that starts the event depending on what are
 them: closers or openers.

 How events are stored:
 Each door block has 256 bytes, one per event line. Each event line is
 located in an offset that is the event line ID, so event line 30 is
 located in the byte 30 of each block.
 There is a door I part of an event line and a door II part of it. We'll
 define them as byte I and byte II.
 You can find there: the door room, the door location, and the
 trigger-next flag. The format is the following:

 Let's define:
  Screen as S and it is a number from 1 to 24 (5 bits)
   S = s1 s2 s3 s4 s5
    where sn is the bit n of the binary representation of S
  Location as L and is a number from 0 to 29 (5 bits)
   L = l1 l2 l3 l4 l5
    where ln is the bit n of the binary representation of L
   This number is according to the location format specifications.
  Trigger-next as T and is a 1 for "off" or a 0 for "on" (1 bit)
   T = t1

 Byte I  has the form: t1 s4 s5 l1 l2 l3 l4 l5
 Byte II has the form: s1 s2 s3  0  0  0  0  0

4.5. Digital Waves
 Read them as raw digital wave sound using the following specifications:

                   Table 4.4: Wave Specifications
                   ~~~~~~~~~~~~~~~~~~~~~~~~~~~~~~
  Size of Format: 16
  Format:         PCM
  Attributes:     8 bit, mono, unsigned
  Channels:       1
  Sample rate:    11025
  Bytes/Second:   11025
  Block Align:    1

 GNU/Linux users can play the raw waves just dropping them inside /dev/dsp
 As dat headers are very small it is valid to type in a shell console with
 dsp write access: cat digisnd?.dat>/dev/dsp to play the whole wave files.

4.6. Midi music
 Standard midi files

4.7. Internal PC Speaker
 We are not so sure about it, but we think it is:
  2 unique bytes for headers
  3 bytes per note (2 for frequency and 1 for duration)

4.8. Binary files
 Some binary files contains relevant information
 The resource number 10 in prince.dat has the VGA guard palettes in it
 saving n records of a 16-colour-palette of 3 bytes in the specified
 palette format.


5. DAT v2.0 Format Specifications
   ~~~ ~~~~ ~~~~~~ ~~~~~~~~~~~~~~

5.1. General file specs, index and checksums
 POP2 DAT files aren't much different from their POP1 predecessors.
 The format is similar in almost each way. The main difference is in the
 index. As DAT v1.0 used an index in the high data, the DAT v2.0 indexes
 are two level encapsulated inside a high data. So there is an index of
 indexes.

 We will use the same conventions than in the prior chapter.
 The checksum validations are still the same.

 High data structures:

 The DAT header: Size = 6 bytes
  - Offset 0, size 4, type UL: HighDataOffset
           (the location where the highData begins)
  - Offset 4, size 2, type US: HighDataSize
           (the number of bytes the highData has)
           Note that HighDataOffset+HighDataSize=file size

 This is similar to DAT v1.0 format, except that the index area is now
 called high data.

 The high data part of the file contains multiple encapsulated indexes.
 Each of those index is indexed in a high data index of indexes. We will
 call this index the �master index� and the sub index the �slave indexes�.
 Slave indexes are the real file contents index.

5.1.1 The master index
 The master index is made with:
  - Offset HighDataOffset,   size 2, type US: NumberOfSlaveIndexes
           (the number of the high data sections)
  - Offset HighDataOffset+2, size NumberOfSlaveIndexes*6: The master index
           record (a list of NumberOfSlaveIndexes blocks of 6-bytes-length
           index record each corresponding to one slave index)

 The 6-bytes-length index record (one per item): Size = 6 bytes
  - Relative offset 0, size 4, type sting: 4 ASCII bytes string denoting
           the section ID. The character order is inverted.
  - Relative offset 4, size 2, type US: SlaveIndexOffset
           (slave index offset relative to HighDataOffset)

 From the end of the DAT High Data index to the end of the file there is
 the High Data section contents (where the HighDataOffset relative offsets
 points to).

 There are different 4 bytes ASCII strings section IDs. When the string is
 less than 4 bytes, they are ended in hex 0x00 is used. We will denote it
 with the cardinal # symbol. The character order is inverted, so for
 example the text SLAP becomes PALS, MARF becomes FRAM, #### becomes empty
 or RCS# becomes SCR. They must be in upper case.

                   Table 5.1: Section ID strings
                   ~~~~~~~~~~~~~~~~~~~~~~~~~~~~~

   ID   Stored Description
   ~~   ~~~~~~ ~~~~~~~~~~~
   cust TSUC   Custom 
   font TNOF   Fonts
   fram MARF   Frames
   palc CLAP   CGA Palette
   pals SLAP   SVGA Palette
   palt TLAP   TGA Palette
   piec CEIP   Pieces  
   psl  LSP#   ? 
   scr  RCS#   Screens (images that have the full room)
   shap PAHS   Shapes (normal graphics)
   shpl LPHS   Shape palettes
   strl LRTS   Text
   snd  DNS#   Sound
   seqs SQES   Midi sequences
   txt4 4TXT   Text
        ####   Levels

5.1.2 The slave indexes
 All encapsulated sections are indexes.
 The slave index is made with:
  - Offset SlaveIndexOffset,   size 2, type US: NumberOfItems
           (the number of the records referring to the file data)
  - Offset SlaveIndexOffset+2, size NumberOfItems*11: The slave index
           record (a list of NumberOfItems blocks of 11-bytes-length index
           record each corresponding to one slave index)

 The 11-bytes-length slave index record (one per item): Size = 11 bytes
  - Relative offset 0, size 2, type US: Item ID
  - Relative offset 2, size 4, type UL: Resource start
           (absolute offset in file)
  - Relative offset 6, size 2, type US: Size of the item
           (not including the checksum byte)
  - Relative offset 8, size 3, type binary: A flags mask
           (in SHAP indexes it's always 0x40 0x00 0x00;
           in others 0x00 0x00 0x00)


5.2. Levels

 This table has a summary of the blocks to be used in this section,
 you can refer it from the text below.

                   Table 5.2: DAT 2.0 Level blocks
                   ~~~~~~~~~~~~~~~~~~~~~~~~~~~~~~~

  Length Offset  Block Name
  ~~~~~~ ~~~~~~  ~~~~~~~~~~
  960    0       wall
  3840   960     pop2_background
  1280   4800    pop2_doors
  128    6080    links (as explained in section 4.4.3 but having 32 rooms)
  32     6208    unknown II
  3      6240    start_position (as explained in section 4.4.5)
  4      6243    unknown III (00 01 00 02) (check pop1)
  3712   6247    pop2_static_guard
  1088   9959    pop2_dynamic_guard
  978    11047   unknown IV


 All levels have a size of 12025.

5.2.1 Room mapping
 You should read section 4.4.2 before reading this one.
 A POP2 level can store a maximum of 32 rooms of 30 tiles each, having 
 three stages of 10 tiles each. Rooms are numbered from 1 to 32 (not 0 to 
 31) because the 0 is be reserved to the null-room.

 The wall block has 32 sub-blocks inside. Each sub-block has a size of 30
 bytes and has a room associated. For each byte in this room there is a
 tile in the game. Each byte has a code to represent a tile. There are
 additional attributes to this tile also.
 
 To locate the 7th tile in the bottom floor of the room 27 you have to do
 the same calculation as in 4.4.2:
  tile=(room-1)*30+tileOffset=(27-1)*30+2*10+7=807
 

                   Table 5.3: Foreground Walls
                   ~~~~~~~~~~~~~~~~~~~~~~~~~~~

 Dec Hex   Bin     Caverns             Ruins              Temple
 ~~~ ~~~   ~~~     ~~~~~~~             ~~~~~              ~~~~~~
           
 00  0x00  000000  Empty               Empty              Empty
 01  0x01  000001  Floor               Floor              Floor
 02  0x02  000010  Spikes              (?)                Spikes
 03  0x03  000011  Pillar              Pillar             Pillar
 04  0x04  000100  Door                Gate               Gate
 05  0x05  000101  (?)                 Raised Button      Raised Button
 06  0x06  000110  (?)                 Drop Button        Drop Button
 07  0x07  000111  (?)                 Tunnel             (?)
 08  0x08  001000  Bottom Big Pillar   Bottom Big Pillar  Bottom Big Pillar
 09  0x09  001001  Top Big Pillar      Top Big Pillar     Top Big Pillar
 10  0x0A  001010  Potion              Potion             Potion
 11  0x0B  001011  Loose Floor         Loose Floor        Loose Floor
 12  0x0C  001100  (?)                 Slicer Left Half   Slicer Left Half
 13  0x0D  001101  (?)                 Slicer Right Half  Slicer Right Half
 14  0x0E  001110  Debris              Debris             Debris
 15  0x0F  001111  (?)                 Drop Floor         (?)
 16  0x10  010000  Exit Half Left      Exit Half Left     Exit Half Left
 17  0x11  010001  Exit Half Right     Exit Half Right    Exit Half Right
 18  0x12  010010  Magic Carpet        (?)                (?)
 19  0x13  010011  Torch               (?)                Torch
 20  0x14  010100  Wall                Wall               Wall
 21  0x15  010101  (?)                 Skeleton           (?)
 22  0x16  010110  (?)                 Sword              (?)
 23  0x17  010111  Lava Pit Left       (?)                (?)
 24  0x18  011000  Lava Pit Right      (?)                (?)
 25  0x19  011001  (?)                 (?)                Squash Wall
 26  0x1A  011010  (?)                 (?)                Flip Tile
 27  0x1B  011011  (?)                 (?)                (?)
 28  0x1C  011100  (?)                 (?)                (?)
 29  0x1D  011101  (?)                 (?)                (?)
 30  0x1E  011110  (?)                 (?)                (?)
 31  0x1F  011111  (?)                 (?)                (?)
 32  0x20  100000  Torch w/Debris      (?)                Torch w/Debris
 33  0x21  100001  Exit Door Top Left  (?)                (?)
 34  0x22  100010  Pressure Plate      (?)                (?)
 35  0x23  100011  Exit Door Top Right (?)                (?)
 36  0x24  100100  Dart Gun            (?)                (?)
 37  0x25  100101  (?)                 (?)                (?)
 38  0x26  100110  (?)                 (?)                (?)
 39  0x27  100111  (?)                 (?)                (?)
 40  0x28  101000  (?)                 (?)                (?)
 41  0x29  101001  (?)                 (?)                (?)
 42  0x2A  101010  (?)                 (?)                (?)
 43  0x2B  101011  (?)                 (?)                Blue Flame
 44  0x2C  101100  Rope Bridge         (?)                (?)
 45  0x2D  101101  (?)                 (?)                (?)
 46  0x2E  101110  (?)                 (?)                (?)
 47  0x2F  101111  (?)                 (?)                (?)

 The pop2_background is an expansion if the pop1_background and it is sized
 4 times bigger. For each tile there are 4 additional bytes in the 
 pop2_background block to specify further actions or attributes. This block
 is sized 4 bytes/tile * 10 tiles/floor * 3 floors/room * 32 rooms that is
 3840 bytes.
 We call background mask to each block of 4 bytes associated to a tile. To
 locate a background mask you have to do the following operation:
  960+(room-1)*30*4+tileOffset*4
 Background masks are stored consecutively each after another until the
 960 tiles are specified.
 
 The first byte is an unsigned char (UC) association to one of the 256 door
 event registers (see section 5.2.2) if the tile is an activator.
 In any other case this byte is an extra attribute information byte.
 For example in wall (0x14) having this byte in 0x04 means the wall is
 curved.
 
 The second byte in a background mask is the attribute byte. For example
 0x18 modifies the tile 0x01 and adds two small stalactites.
 
 We believe the special images uses the 3rd or 4th byte.

5.2.2 Door events
 This section explains how doors are handled and specifies the block
 pop2_door.

 The pop2_door block has 1280 bytes. It is divided in 256 registers of
 5 bytes called door events. Like pop1 events have associations to doors
 and activate them. In POP2 events can also activate a floor shooter.
 
 An event is triggered when an activator button (0x22) is pressed. As it is
 specified in the section 5.2.1, the first byte of the attribute mask
 belonging to a button tile points it to a door event that is triggered
 when the button is pressed.
 There is a maximum of 256 events because of the unsigned char of the first
 byte if the attribute mask in the pop2_background block and the 256
 registers in the pop2_door block.

 Each event register is of the form "LL SS TT FD FD" which activates the
 normal door (0x04), right exit door (0x11) or shooter (0x24) located in
 the tile LL of the screen SS. TT is 00 for normal activation and FF for
 exit doors.

5.2.3 Guard handling
 This section explains how guards are handled. In POP2 there are two
 different types of guards. We'll call them static and dynamic guards.
 Static guards are the normal guards that are waiting in a room like in
 POP1. In the other hand, dynamic guards are the ones who appear in the
 room running from one of the sides. Each type of guard is attached to a
 room and is handled in a different way, so a room can have both types of
 guards, one or none depending on the specifications. There is a block for
 each type of guard, pop2_static_guard is the specification of the static
 guards and pop2_dynamic_guard is the specification of the dynamic ones.
 Each block has different specifications and sizes as it is mentioned
 bellow.
 
5.2.3.1 Static guards
 
 In this item static guards are explained and the pop2_static_guard is
 specified.
 
 For each screen there is reserved memory space for a maximum of 5 guards.
 
 The pop2_static_guard block has a size of 3712 divided in 32 sub-blocks of
 116 bytes each. As there is a correspondence between each sub-block and
 the room with this number, we'll call them "room guard blocks".
 
 A room guard block has a size of 116 divided this way:
 - 1 byte for the number of guards present in this room.
   This byte may take values from 0 to 5.
 - 5 block divisions of 23 bytes for each guard.
   The first divisions have the guard information, if the number is less
   than 5, then the latest divisions corresponding to the missing guards
   will be filled with zeros or garbage.
 
 If there is a static guard corresponding to this division of 23 bytes:
 
 From 0 to 22 the known bytes are:
 
 Byte 0 is a UC showing the location in this room (same as explained in
  4.4.4) of the current guard.
 Bytes 1 and 2 are a SS with an offset in pixels to reallocate the guard in
  the floor.
 Byte 3 is the facing direction as specified in 4.4.4.
 Byte 4 is the skill
 Byte 9 is the guard colour in levels where it is needed (eg. 1 white+blue,
  2 white+black, 3 red)
 Byte 15 is the type (eg. 5 head, 8 snake), doesn't apply in all levels
 Byte 16 is the hit points if the guard (0 to 8)

 Bytes from 5 to 8 and 17 to 22 are still unknown.

5.2.3.2 Dynamic guards
 
 The dynamic guards are the ones who appear running throw a room's corner
 and they are defined in the dynamic_guard_block.
 This block has 34 bytes for each of the 32 rooms, so it is sized 1088
 bytes. Each room has a specification about those guards.
 There is only one different type of dynamic guard per room, but it is
 possible to set the number of guards that will appear running.
 
 The bytes are from 0 to 33:
 Byte 18 activates dynamic guard. 1 is true and 0 is false.
 Byte 23 is where the guard is located
 Byte 24 is the floor the guard will appear on 0 is the upper one and 2 is
  the lower. Another number will kill the guard playing the sound.
 Byte 30 is the number of guards that will appear.


6. PLV v1.0 Format Specifications
   ~~~ ~~~~ ~~~~~~ ~~~~~~~~~~~~~~

 PLV v1.0 files are defined in this table:

                   Table 6.1: PLV blocks
                   ~~~~~~~~~~~~~~~~~~~~~

   Size Offset Description                  Type   Content
   ~~~~ ~~~~~~ ~~~~~~~~~~~                  ~~~~   ~~~~~~~
      7      0 Magic identifier             text   "POP_LVL"
      1      7 POP version                  UC     0x01
      1      8 PLV version                  UC     0x01
      1      9 Level Number                 UC
      4     10 Number of fields             UL
      4     14 Block 1: Level size (B1)     UL     2306/2305
     B1     18 Block 1: Level code          -
      4  18+B1 Block 2: User data size (B2) UL
     B2  22+B1 Block 2: User data           -

 Level code is the exact level as described in 4.4 including the checksum
 byte. Note that Level size (B1)  also includes the checksum byte in the
 count.
 POP version is 1 for POP1 and 2 for POP2.
 PLV version is 1 for PLV v1.0.
 Only one level may be saved in a PLV, the level number is saved inside.

6.1. User data

 User data is a block of extensible information, Number of fields is the
 count of each field/value information pair. A pair is saved in the
 following format:
  field_name\0value\0
 where \0 is the null byte (0x00) and field_name and value are strings.

 There are mandatory pairs that must be included in all PLV files.
 Those are:

                   Table 6.2: Mandatory Fields
                   ~~~~~~~~~~~~~~~~~~~~~~~~~~~

  Field name              Description
  ~~~~~~~~~~              ~~~~~~~~~~~
  Editor Name             The name of the editor used to save the file
  Editor Version          The version of the editor used to save the file
  Level Author            The author of the file
  Level Title             A title for the level
  Level Description       A description
  Time Created            The time when the file was created
  Time Last Modified      The time of the last modification to the file
  Original Filename       The name of the original file name (levels.dat)
  Original Level Number   Optional. The level number it has when it was
                          first exported
 
 The content values may be empty. There is no need to keep an order within
 the fields.

6.2. Allowed Date format
 To make easy time parsing the time format must be very strict.
 There are only two allowed formats: with seconds and without.
 With seconds the format is "YYYY-MM-DD HH:II:SS"
 Without seconds the format is "YYYY-MM-DD HH:II"
 Where YYYY is the year in 4 digits, MM is the month in numbers, MM the
 months, DD the days, HH the hour, II the minute and SS the second in the
 military time: HH is a number from 00 to 23.

 If the month, day, hour or second have only one digit, the other digit
 must be completed with 0.
 i.e. 2002-11-26 22:16:39


7. The SAV v1.0 format
   ~~~ ~~~ ~~~~ ~~~~~~

 SAV v1.0 saves kid level, lives and remaining time information in order to
 restart the game from this position.

 SAV files are 8 bytes length in the following format:

                   Table 7.1: SAV blocks
                   ~~~~~~~~~~~~~~~~~~~~~

   Size Offset Description                  Type
   ~~~~ ~~~~~~ ~~~~~~~~~~~                  ~~~~
      2      0 Remaining minutes            US   (i)
      2      2 Remaining ticks              US   (ii)
      2      4 Current level                US   (iii)
      2      6 Current hit points           US   (iv)

 Remaining minutes (i)
  Range values:
   0     to 32766 for minutes
   32767 to 65534 for NO TIME (but the time is stored)
   65535 for game over

 Remaining ticks (ii)
  Seconds are stored in ticks, a tick is 1/12 seconds. To get the time in
  seconds you have to divide the integer "Remaining ticks" by 12.

  Range values:
   0.000 to 59.916 seconds
                   (rounded by units of 83 milliseconds or 1/12 seconds)
   0     to 719    ticks

 Level (iii)
  Range values:
   1  to 12 for normal levels
   13 for 12bis
   14 for princess level
   15 for potion level

 Hit points (iv)
  Range values:
   0 for an immediate death
   1 to 65535 lives


8. The HOF v1.0 format
   ~~~ ~~~ ~~~~ ~~~~~~
 HOF files are used to save the Hall of Fame information.

 All HOF v1.0 files have a size of 176 bytes. The first 2 bytes belongs to
 the record count. The format is US. The maximum number of records allowed
 is 6, so the second byte is always 0x00.
 Following those bytes there is an array of records. This array has a full
 size of 29 bytes distributed according to the following table.

                   Table 8.1: HOF blocks
                   ~~~~~~~~~~~~~~~~~~~~~

   Size Offset Description                  Type
   ~~~~ ~~~~~~ ~~~~~~~~~~~                  ~~~~
     25      0 Player name                  text
      2     25 Remaining minutes            US (similar to SAV format)
      2     27 Remaining ticks              US (similar to SAV format)

 In case there is no record, the 29 bytes spaces must be filled with zeros
 in order to complete the whole file and give it the size of 2+29*6 = 176.


9. Credits
   ~~~~~~~

 This document:
  Writing . . . . . . . . . . . . . . . . . . . . . . . . . Enrique Calot
  Corrections . . . . . . . . . . . . . . . . . . . . .  Patrik Jakobsson

 Reverse Engineering:
  Indexes . . . . . . . . . . . . . . . . . . . . . . . . . Enrique Calot
  Levels . . . . . . . . . . . . . . . . . . . . . . . . .  Enrique Calot
                                                            Brendon James
  Images . . . . . . . . . . . . . . . . . . . . . . .  Tammo Jan Dijkema
  RLE Compression . . . . . . . . . . . . . . . . . . . Tammo Jan Dijkema
  LZG Compression . . . . . . . . . . . . . . . . . . . . . Anke Balderer
                                                             Diego Essaya
  Sounds . . . . . . . . . . . . . . . . . . . . . . . Christian Lundheim

 PLV v1.0:
  Definition . . . . . . . . . . . . . . . . . . . . . . .  Brendon James
                                                            Enrique Calot

10. License
    ~~~~~~~

      Copyright (c)  2004, 2005, 2006 The Princed Project Team
      Permission is granted to copy, distribute and/or modify this document
      under the terms of the GNU Free Documentation License, Version 1.2
      or any later version published by the Free Software Foundation;
      with no Invariant Sections, no Front-Cover Texts, and no Back-Cover
      Texts.  A copy of the license is included in the section entitled
      "GNU Free Documentation License".

